% Options for packages loaded elsewhere
\PassOptionsToPackage{unicode}{hyperref}
\PassOptionsToPackage{hyphens}{url}
%
\documentclass[
]{article}
\title{ACLIM2 ESR indices}
\author{K. Holsman}
\date{}

\usepackage{amsmath,amssymb}
\usepackage{lmodern}
\usepackage{iftex}
\ifPDFTeX
  \usepackage[T1]{fontenc}
  \usepackage[utf8]{inputenc}
  \usepackage{textcomp} % provide euro and other symbols
\else % if luatex or xetex
  \usepackage{unicode-math}
  \defaultfontfeatures{Scale=MatchLowercase}
  \defaultfontfeatures[\rmfamily]{Ligatures=TeX,Scale=1}
\fi
% Use upquote if available, for straight quotes in verbatim environments
\IfFileExists{upquote.sty}{\usepackage{upquote}}{}
\IfFileExists{microtype.sty}{% use microtype if available
  \usepackage[]{microtype}
  \UseMicrotypeSet[protrusion]{basicmath} % disable protrusion for tt fonts
}{}
\makeatletter
\@ifundefined{KOMAClassName}{% if non-KOMA class
  \IfFileExists{parskip.sty}{%
    \usepackage{parskip}
  }{% else
    \setlength{\parindent}{0pt}
    \setlength{\parskip}{6pt plus 2pt minus 1pt}}
}{% if KOMA class
  \KOMAoptions{parskip=half}}
\makeatother
\usepackage{xcolor}
\IfFileExists{xurl.sty}{\usepackage{xurl}}{} % add URL line breaks if available
\IfFileExists{bookmark.sty}{\usepackage{bookmark}}{\usepackage{hyperref}}
\hypersetup{
  pdftitle={ACLIM2 ESR indices},
  pdfauthor={K. Holsman},
  hidelinks,
  pdfcreator={LaTeX via pandoc}}
\urlstyle{same} % disable monospaced font for URLs
\usepackage[margin=1in]{geometry}
\usepackage{graphicx}
\makeatletter
\def\maxwidth{\ifdim\Gin@nat@width>\linewidth\linewidth\else\Gin@nat@width\fi}
\def\maxheight{\ifdim\Gin@nat@height>\textheight\textheight\else\Gin@nat@height\fi}
\makeatother
% Scale images if necessary, so that they will not overflow the page
% margins by default, and it is still possible to overwrite the defaults
% using explicit options in \includegraphics[width, height, ...]{}
\setkeys{Gin}{width=\maxwidth,height=\maxheight,keepaspectratio}
% Set default figure placement to htbp
\makeatletter
\def\fps@figure{htbp}
\makeatother
\setlength{\emergencystretch}{3em} % prevent overfull lines
\providecommand{\tightlist}{%
  \setlength{\itemsep}{0pt}\setlength{\parskip}{0pt}}
\setcounter{secnumdepth}{-\maxdimen} % remove section numbering
\usepackage{booktabs}
\usepackage{longtable}
\usepackage{array}
\usepackage{multirow}
\usepackage{wrapfig}
\usepackage{float}
\usepackage{colortbl}
\usepackage{pdflscape}
\usepackage{tabu}
\usepackage{threeparttable}
\usepackage{threeparttablex}
\usepackage[normalem]{ulem}
\usepackage{makecell}
\usepackage{xcolor}
\ifLuaTeX
  \usepackage{selnolig}  % disable illegal ligatures
\fi

\begin{document}
\maketitle

\hypertarget{high-resolution-climate-change-projections-for-the-eastern-bering-sea}{%
\subsection{High resolution climate change projections for the Eastern
Bering
Sea}\label{high-resolution-climate-change-projections-for-the-eastern-bering-sea}}

Kirstin K. Holsman\(^1\),Albert Hermann\(^2\), Wei Cheng\(^2\), Kelly
Kearney\(^2\), Darren Pilcher\(^2\), Kerim Aydin\(^1\), Ivonne
Ortiz\(^2\)

\href{mailto:kirstin.holsman@noaa.gov}{\nolinkurl{kirstin.holsman@noaa.gov}}

\(^1\)Cooperative Institute for Climate, Ocean and Ecosystem Studies,
University of Washington, Seattle, WA. 98195

\(^2\)Alaska Fisheries Science Center, NOAA, 7600 Sand Point Way N.E.,
Bld. 4, Seattle, Washington 98115

\textbf{Last Updated: October 2022}

\hypertarget{summary-statement}{%
\subsection{Summary statement:}\label{summary-statement}}

A high resolution oceanographic model for the Bering Sea projects
widespread warming across the region under low carbon mitigation (high
greenhouse gas emission) scenarios (`ssp585' for Shared Socioeconomic
Pathway 5-8.5) with modeled bottom temperatures exceeding historical
ranges by mid-century onward. High carbon mitigation scenarios (`ssp126'
for Shared Socioeconomic Pathway 1-2.6) are also associated with slight
warming over the next century but modeled end-of-century temperatures do
not exceed historical ranges (1980-2013) as estimated by the hindcast.

Global Warming Levels (GWL) of +3 and +4 \(^o\)C over pre-industrial
temperatures (1850-1900) are associated with significant warming of
surface and bottom water temperatures in both the Northern Bering and
Southern Bering Seas (NEBS and SEBS, respectively). In contrast, under
GWL of +1.5 \(^o\)C, SST and BT in both regions remain within the
near-present range of climate variability (2010-2021). Global Warming
Levels of +1.5 and +2 \(^o\)C represent the target and limit
respectively of the UNFCCC
\href{https://unfccc.int/ndc-synthesis-report-2022\#Projected-GHG-Emission-levels}{Paris
Agreement and Nationally determined contributions (NDCs)}; UNFCCC 2022)
and are the warming levels beyond which climate change impacts and risks
increase, --and feasibility and effectiveness of adaptation actions
decrease--, rapidly with each GWL (IPCC 2022).

Warming of modeled bottom temperatures across seasons and models is
projected in the SEBS under both mitigation scenarios, and in the NEBS
under low carbon mitigation scenarios (ssp585) and is generally larger
across regions and seasons under low carbon mitigation scenarios. While
NEBS bottom water warming was also projected for summer months under
high carbon mitigation scenarios (ssp126), only 1 of the 3 models
projected substantial warming during winter months, indicating potential
for sea ice and cold bottom water temperatures to be preserved to some
extent through the end of century under high carbon mitigation scenarios
(ssp126).

\hypertarget{status-and-trends}{%
\subsection{Status and trends:}\label{status-and-trends}}

Summer bottom temperatures in both the Southern Bering Sea (SEBS) and
the Northern Bering Sea (NEBS) are projected to increase overtime, with
higher rates of warming associated with low carbon mitigation (higher
greenhouse gas emission) scenarios (ssp585) relative to high carbon
mitigation scenarios (ssp126; Figs. 1, 2). Three Earth Systems Models
(ESMs) are presented to reflect the spread in projections across
ensemble members (see Description of Index below). There is general
agreement in all three models (ESMs) with respect to trends in warming
associated with alternative climate scenarios. For the SEBS, estimates
of end of century warming of bottom temperatures
{[}(2080-2100)-(1980-2013){]} range from +0.04 to +2.51 \(^o\)C for high
carbon mitigation (ssp126) scenarios and +2.05 to +4.17 \(^o\)C for low
carbon mitigation (ssp585) scenarios (ranges represent +/- 1 standard
deviation; Fig. 1). For the NEBS, estimates of end of century warming of
bottom temperatures {[}(2080-2100)-(1980-2013){]} range from +0.07 to
+3.01 \(^o\)C for high carbon mitigation (ssp126) scenarios and +2.82 to
+6.58 \(^o\)C for low carbon mitigation (ssp585) scenarios(Fig. 2). In
high emission scenarios, bottom temperatures for the SEBS and NEBS by
the mid-century (2050-2060) are projected to consistently (i.e., all
three ensemble members) exceed the upper range of historical modeled
temperatures from the Bering Sea shelf summer trawl survey and hindcast
simulations. Mean historical (1980-2013) bottom temperatures from the
hindcast are 3.2 (SD = 0.76 ) and 2.65 (SD = 0.98 ) for the SEBS and
NEBS, respectively.

Model projected trends for sea surface temperature (SST) are similar to
those of bottom temperature but are warmer and with higher agreement for
all three models under low carbon mitigation scenarios (ssp585). Under
low carbon mitigation scenarios, estimates of end of century warming of
SST {[}(2080-2100)-(1980-2013){]} range from +0.65 to +3.02 \(^o\)C for
high carbon mitigation (ssp126) scenarios and +3.05 to +5.09 \(^o\)C for
low carbon mitigation (ssp585) scenarios. For the NEBS, estimates of end
of century warming of SST {[}(2080-2100)-(1980-2013){]} range from +0.72
to +3.88 \(^o\)C for high carbon mitigation (ssp126) scenarios and +4.03
to +6.69 \(^o\)C for low carbon mitigation (ssp585) scenarios (Fig. 2).
Mean historical (1980-2013) SSTs from the hindcast are 9.7 (SD = 0.8 )
and 8.34 (SD = 1.01 ) for the SEBS and NEBS, respectively.

Global Warming Levels (GWL) of +3 and +4 \(^o\)C over pre-industrial
temperatures (1850-1900) are associated with significant warming of
surface and bottom water temperatures in both the Northern Bering and
Southern Bering Seas (NEBS and SEBS, respectively; Fig. 3). In contrast,
under GWL of +1.5 \(^o\)C, SST and BT in both regions remain within the
near-present range of climate variability (2010-2021). GWL of +1.5 and
+2 \(^o\)C represent the target and limit respectively of the UNFCCC
\href{https://unfccc.int/ndc-synthesis-report-2022\#Projected-GHG-Emission-levels}{Paris
Agreement and Nationally determined contributions (NDCs)}; UNFCCC 2022),
i.e.~the GWLs beyond which climate change impacts and risks increase,-
and feasibility and effectiveness of adaptation actions decrease-,
rapidly with each GWL (IPCC 2022). As a point of reference, in the most
recent 6th assessment report the IPCC found that 2019 GWL was +1.08
\(^o\)C (IPCC 2021).

Both the SEBS and NEBS exhibit seasonal patterns in BT and SST, and
seasonally-specific patterns in warming (relative to 1980-2013
climatology from corresponding historical runs for each ESM) across IPCC
scenarios and among ESM models (Figs. 4, 5; note these plots are of
non-bias corrected values). In general there is agreement in warming
trends among the three ensemble members for all seasons in the SEBS,
especially under low carbon mitigation (high greenhouse gas emissions;
ssp585) scenarios (Fig. 4), although the magnitude varies with model
(cesm \textgreater{} miroc \textgreater{} gfdl). In the NEBS there is
more variability among the three ensemble members under high mitigation
scenarios (ssp126), with 2 of 3 models projecting little warming in
winter months(exception is cesm). However, under low carbon mitigation
(high greenhouse gas emissions) scenarios (ssp585), all three models
project warming in winter months (i.e., reduced sea ice) as well as
increases in spring, summer, and fall BT (Fig. 5).

\hypertarget{factors-influencing-observed-trends}{%
\subsection{Factors influencing observed
trends}\label{factors-influencing-observed-trends}}

For more information about climate change impacts, risks, adaptation,
and mitigation see the Intergovernmental Panel on Climate Change (IPCC)
\href{https://www.ipcc.ch/assessment-report/ar6/}{6th Assessment Report
(www.ipcc.ch/assessment-report/ar6/)} and \url{www.climate.gov}.

Carbon dioxide (CO\(_2\)) is naturally occurring greenhouse gas (GHG)
that along with other GHGs acts to absorb and re-emit infrared energy
(heat) from solar radiation, warming the earth's surface (i.e., the
`greenhouse effect'). Naturally occurring CO\(_2\) (ocean off-gassing,
volcanoes, etc.) is naturally offset by carbon sinks (e.g.,
photosynthesis of plants on land and in the ocean) acting to keep
CO\(_2\) relatively stable for more than 800,000 years at or below 300
parts per million (ppm). However, scientific observations and models
have shown that atmospheric CO\(_2\) concentrations have been rising
steadily over the past century due to anthropogenic (human) activities,
primarily the the burning of fossil fuels for energy and other uses.
Rates of anthropogenic CO\(_2\) release into the atmosphere exceed
natural carbon sinks and have resulted in a rapid accumulation of
atmospheric CO\(_2\). The IPCC 6th Assessment Report states that
\emph{``observed increases in well-mixed greenhouse gas (GHG)
concentrations since around 1750 are unequivocally caused by human
activities''} and that the \emph{``land and ocean have taken up a
near-constant proportion (globally about 56\% per year) of CO\(_2\)
emissions from human activities over the past six decades, with regional
differences (high confidence)''} (IPCC 2021).

Current atmospheric CO\(_2\) levels of 410 ppm (IPCC 2021) have not been
experienced for at least 2 million years, and the rate of increase in
CO\(_2\) over the last century is unprecedented in the last 800,000
years (based on multiple lines of evidence including Antarctic ice core
data and isotopes IPCC 2021). There is a near-linear relationship
between cumulative CO\(_2\) emissions and increases in global surface
temperature (IPCC 2021), and changes in atmospheric CO\(_2\) and
associated warming have direct impacts on ocean processes and chemistry.
In the most recent assessment report, the IPPC (2021) states
\emph{``better integration of paleo-oceanographic data with modelling
along with higher-resolution analyses of transient changes have improved
understanding of long-term ocean processes\ldots This paleo context
supports the assessment that ongoing increase in ocean heat content
(OHC) represents a long-term commitment, essentially irreversible on
human time scales (high confidence)''}. This paleo context has also
helped illuminate the complex role of oceans in the regulation of the
global climate and atmospheric CO\(_2\) during previous
glacial--interglacial warming intervals. Presently, absorption of
atmospheric heat by the world's oceans increases ocean temperatures,
warming surface waters to depth and raising the ``ocean heat content''.
Absorption of atmospheric CO\(_2\) alters the chemistry of the ocean
increasing acidity and lowering the pH. In addition, atmospheric warming
alters physical and chemical processes (e.g., precipitation, wind
patterns, sea level, ocean circulation, and sea ice thickness and
extent) in ways that further change the ocean and atmospheric cycles,
i.e., the climate of a given region. Accordingly, the IPCC (2021)
states, ``it is \emph{virtually certain} that the global upper ocean
(0--700 m) has warmed since the 1970s and \emph{extremely likely} that
human influence is the main driver. It is \emph{virtually certain} that
human-caused CO\(_2\) emissions are the main driver of current global
acidification of the surface open ocean''.

The near linear relationship between cumulative CO\(_2\) emissions and
increases in global surface temperature (IPCC 2021) has enabled
scientists evaluate future climate conditions under alternative CO\(_2\)
and GHG emissions scenarios, known as Shared Socioeconomic Pathways
(SSPs; O'neil et al.~2017). These allow for projections of changes in
climate and ocean temperature and chemistry under Global Warming Levels
(GWLs). Patterns of warming reported in this contribution reflect global
changes in atmospheric carbon, climate conditions, and oceanic
conditions from Earth Systems Models, but are refined through a regional
lens via a the high resolution Bering10K ROMSNPZ ocean model that is
able to replicate fine scale oceanographic processes (e.g., changes in
sea ice and circulation on a short timestep) that act to amplify or
attenuate larger scale climate change effects.

\hypertarget{implications}{%
\subsection{Implications:}\label{implications}}

Historically, warming temperatures and marine heat waves have been
associated with changes to food-web dynamics, species redistribution,
and ecosystem structure and processes (Huntington et al.~2020).
Projected ocean warming from global models is associated with declines
in marine fish biomass, benthic biomass, and fisheries catch potential
(IPCC AR6 WGII). Evaluations of projected temperature effects in Bering
sea ecosystems and fisheries under high emission scenarios (ssp585) are
still underway but include modeled declines in winter sea ice and summer
cold pool extent associated with increased warming under low carbon
mitigation scenarios (ssp585). Increased warming in EBS projections is
also associated with emergent declines in modeled fall euphausiid and
large copepod biomass (Hermann et al.~2021), shifts in spring bloom
timing to earlier (30-60 d) and slightly larger phytoplankton and
zooplankton blooms (relative to hindcasts), and declines in the
magnitude of fall total phytoplankton and large zooplankton blooms
(Cheng et al.~2021).

\hypertarget{description-of-index}{%
\subsection{Description of index:}\label{description-of-index}}

We report trends in modeled bottom temperature and sea surface
temperature from a 30-layer Bering Sea regional oceanographic model at
10km horizontal resolution which has incorporated lower trophic level
biology (Kearney et al.~2020) and marine carbonate chemistry (Pilcher et
al.~2019). See the
\href{https://zenodo.org/record/4586950/files/Bering10K_dataset_documentation.pdf}{\emph{Bering
10K dataset documentation}} for more information and technical details.
We present the
\href{www.integratedecosystemassessment.noaa.gov}{\emph{Alaska NOAA
Integrated Ecosystem Assessment (IEA) program}} annual hindcast and two
CMIP6 emission scenarios projected as part of the
\href{www.fisheries.noaa.gov/alaska/ecosystems/alaska-climate-integrated-modeling-project}{\emph{Alaska
Climate Integrated Modeling (ACLIM) project}}. In this, a low
atmospheric carbon emissions scenario (ssp126) and a low carbon
mitigation scenario (ssp585; O'Neil et
al.~\href{https://link.springer.com/article/10.1007/s10584-013-0905-2}{2017})
and three global Earth System Models (ESMs; `cesm', `gfdl' and `mir')
were selected from the
\href{https://www.wcrp-climate.org/wgcm-cmip/wgcm-cmip6}{Coupled Model
Intercomparison Project (CMIP6)} and used to force the regional model.
See Hermann et al.~2021, Cheng et al.~2021, Kearney et al.~2020, and
Pilcher et al.~2019 for details about the regional model projections and
Hollowed et al.~2020 for details about the
\href{www.fisheries.noaa.gov/alaska/ecosystems/alaska-climate-integrated-modeling-project}{ACLIM}
project and forcing (climate scenario and ESM) selection.

In support of
\href{www.fisheries.noaa.gov/alaska/ecosystems/alaska-climate-integrated-modeling-project}{ACLIM},
a number of different biophysical index timeseries were calculated based
on the Bering10K simulations and provide the primary means of linking
the physical and lower trophic level dynamics to the ACLIM suite of
upper trophic level and socioeconomic models; see Hollowed et al.~2020
for further details. The timeseries reported here are derived from the
area-weighted strata averages for Summer (months Jul - Sep) and Winter
(Jan - Feb) for the Northern Bering Sea (strata 70, 81, 82, 90 of the
eastern Bering Sea shelf bottom trawl survey of the Alaska Fisheries
Science Center) and Southern Bering sea (strata 10, 20, 31, 32, 50, 20,
41, 42, 43, 61, 62). The timerseries were bias-corrected to hindcast
simulations using historical forcing (during 1980-2013) from each ESM.
More detail on this approach is available by request.

The climate simulations presented here are dynamically downscaled from a
selection of the historical and shared socioeconomic pathway simulations
from the sixth phase of the
\href{https://www.wcrp-climate.org/wgcm-cmip/wgcm-cmip6}{CMIP6}. Names
reflect the parent global model simulation (miroc = MIROC ES2L, cesm =
CESM2, gfdl = GFDL ESM4) and emissions scenario via Shared Socioeconomic
Pathways (SSPs) (ssp126 = SSP1-2.6, ssp585 = SSP5-8.5, historical =
Historical). ssp126 represents a high carbon mitigation (low greenhouse
gas emissions) scenario; ssp585 represents the low carbon mitigation
scenario. More information on the SSPs and their use in climate
projections is available at O'Neil et
al.~\href{https://link.springer.com/article/10.1007/s10584-013-0905-2}{2017}.

To determine mean temperatures associated with standardized levels of
global warming for each scenario and ESM, we used CMIP6 Global Warming
Levels from Hauser et
al.~\href{https://doi.org/10.5281/zenodo.4600706}{2021} and publicly
available at \url{https://github.com/mathause/cmip_warming_levels}.

\hypertarget{literature-cited}{%
\subsection{Literature Cited}\label{literature-cited}}

Cheng et al., 2021 W. Cheng, A.J. Hermann, A.B. Hollowed, K.K. Holsman,
K.A. Kearney, D.J. Pilcher, et al.~Eastern Bering Sea shelf
environmental and lower trophic level responses to climate forcing:
results of dynamical downscaling from CMIP6 Deep-Sea Res. II, 193
(2021), Article 104975,
\href{https://www.sciencedirect.com/science/article/pii/S0967064521000515}{10.1016/j.dsr2.2021.104975}.

Hauser, M., F. Engelbrecht, F. \& E. Fischer (2021). Transient global
warming levels for CMIP5 and CMIP6 (v0.2.0) {[}Data set{]}.
\url{https://github.com/mathause/cmip_warming_levels/tree/v0.2.0}.
\href{https://doi.org/10.5281/zenodo.4600706}{doi.org/10.5281/zenodo.4600706}.

Hermann et al., 2021 A.J. Hermann, K. Kearney, W. Cheng, D. Pilcher, K.
Aydin, K.K. Holsman, et al.~Coupled modes of projected regional change
in the Bering Sea from a dynamically downscaling model under CMIP6
forcing Deep-Sea Res. II (2021),
\href{https://www.sciencedirect.com/science/article/pii/S0967064521000503}{10.1016/j.dsr2.2021.104974
194 104974}.

Hollowed, K. K. Holsman, A. C. Haynie, A. J. Hermann, A. E. Punt, K. Y.
Aydin, J. N. Ianelli, S. Kasperski, W. Cheng, A. Faig, K. Kearney, J. C.
P. Reum, P. D. Spencer, I. Spies, W. J. Stockhausen, C. S. Szuwalski, G.
Whitehouse, and T. K. Wilderbuer. Integrated modeling to evaluate
climate change impacts on coupled social-ecological systems in Alaska.
Frontiers in Marine Science, 6(January):1--18, 2020. DOI:
\href{https://www.frontiersin.org/articles/10.3389/fmars.2019.00775/full}{10.3389/fmars.2019.00775}.

IPCC, 2021: Summary for Policymakers. In: Climate Change 2021: The
Physical Science Basis. Contribution of Working Group I to the Sixth
Assessment Report of the Intergovernmental Panel on Climate Change
{[}Masson-Delmotte, V., P. Zhai, A. Pirani, S.L. Connors, C. Péan, S.
Berger, N. Caud, Y. Chen, L. Goldfarb, M.I. Gomis, M. Huang, K.
Leitzell, E. Lonnoy, J.B.R. Matthews, T.K. Maycock, T. Waterfield, O.
Yelekçi, R. Yu, and B. Zhou (eds.){]}. Cambridge University Press,
Cambridge, United Kingdom and New York, NY, USA, pp.~3−32,
\href{https://www.ipcc.ch/report/ar6/wg1/downloads/report/IPCC_AR6_WGI_SPM.pdf}{doi:10.1017/9781009157896.001}.

IPCC, 2022: Climate Change 2022: Impacts, Adaptation and Vulnerability.
Contribution of Working Group II to the Sixth Assessment Report of the
Intergovernmental Panel on Climate Change {[}H.-O. Pörtner, D.C.
Roberts, M. Tignor, E.S. Poloczanska, K. Mintenbeck, A. Alegría, M.
Craig, S. Langsdorf, S. Löschke, V. Möller, A. Okem, B. Rama (eds.){]}.
Cambridge University Press. Cambridge University Press, Cambridge, UK
and New York, NY, USA, 3056 pp.,
\href{https://report.ipcc.ch/ar6/wg2/IPCC_AR6_WGII_FullReport.pdf}{doi:10.1017/9781009325844}.

Kearney, K, A. Hermann, W. Cheng, I. Ortiz, and K. Aydin. A coupled
pelagic benthic-sympagic biogeochemical model for the Bering Sea:
documentation and validation of the BESTNPZ model (v2019.08.23) within a
high resolution regional ocean model. Geoscientific Model Development,
13 (2):597--650, 2020. DOI:
\href{https://www.geosci-model-dev.net/13/597/2020/10\%20https://github.com/beringnpz/roms-bering-sea}{10.5194/gmd13-597-2020}.

O'Neill, B. C.,E. Kriegler, K. Riahi, K. L. Ebi, S. Hallegatte, T. R.
Carter, R. Mathur \& D. P. van Vuuren (2017) A new scenario framework
for climate change research: the concept of shared socioeconomic
pathways. Climatic Change (2014) 122:387--400
\href{https://link.springer.com/article/10.1007/s10584-013-0905-2}{DOI
10.1007/s10584-013-0905-2}

Pilcher, D. J., D. M. Naiman, J. N. Cross, A. J. Hermann, S. A.
Siedlecki, G. A. Gibson, and J. T. Mathis. Modeled Effect of Coastal
Biogeochemical Processes, Climate Variability, and Ocean Acidification
on Aragonite Saturation State in the Bering Sea. Frontiers in Marine
Science, 5(January):1--18, 2019. DOI:
\href{https://www.frontiersin.org/articles/10.3389/fmars.2018.00508/full}{10.3389/fmars.2018.00508}
12.

Pilcher, D. J., J.N.Cross, A.J.Hermanna, K.A.Kearneya, W.Chenga,
J.T.Mathis. 2022. Dynamically downscaled projections of ocean
acidification for the Bering Sea. Deep Sea Res II. (198)
\url{https://doi.org/10.1016/j.dsr2.2022.105055}.

UNFCCC 2022. Nationally determined contributions under the Paris
Agreement Synthesis report by the secretariat. United Nations Framework
Convention on Climate Change. Conference of the Parties serving as the
meeting of the Parties to the Paris Agreement Fourth session. Sharm
el-Sheikh, 6--18 November 2022. FCCC/PA/CMA/2022/4
\url{https://unfccc.int/ndc-synthesis-report-2022}.

\hypertarget{figures}{%
\subsection{Figures:}\label{figures}}

\begin{figure}
\centering
\includegraphics[width=1\textwidth,height=\textheight]{ESR_EBS/Figs/ACLIM_Holsman_Fig1.png}
\caption{Figure 1. Bias corrected summer sea surface temperature (top
row) and bottom temperature (bottom row) for the southern Bering Sea
(SEBS) from the hindcast (dark blue line) and projections under low
(ssp126, left column; cool colors) and high (ssp585, right column, warm
colors) emission scenarios; individual Earth System Models are shown as
individual lines. Average modeled temperatures from the reference period
(1980 -2013) of the hindcast are show as the horizontal blue line;
dashed lines represent +/- 1 standard deviation of the mean. Note
different scales between rows.}
\end{figure}

\begin{figure}
\centering
\includegraphics[width=1\textwidth,height=\textheight]{ESR_EBS/Figs/ACLIM_Holsman_Fig2.png}
\caption{Figure 2, Bias corrected summer sea surface temperature (top
row) and bottom temperature (bottom row) for the Northern Bering Sea
(NEBS) from the hindcast (dark blue line) and projections under low
(ssp126, left column; cool colors) and high (ssp585, right column, warm
colors) emission scenarios; individual Earth System Models are shown as
individual lines. Average modeled temperatures from the reference period
(1980 -2013) of the hindcast are show as the horizontal blue line;
dashed lines represent +/- 1 standard deviation of the mean. Note
different scales between rows.}
\end{figure}

\begin{figure}
\centering
\includegraphics[width=1\textwidth,height=\textheight]{ESR_EBS/Figs/ACLIM_Holsman_Fig3.png}
\caption{Figure 3. Southern and Northern Bering Sea (`SEBS' and `NEBS',
respectively) modeled summer bottom and sea surface temperatures (`BT'
and `SST', respectively) as a function of CMIP6 Global Warming Levels
(mean global increase in temperature relative to pre-industrial
temperatures (1850-1900)). Recent hindcast ranges are reported
(`2010-2021') as well as bias corrected projections from the Bering10K
model for each GWL (+1 to +4 GWL). Boxplots represent the 25th and 75th
percentile (i.e, the interquartile range) with the horizontal line
representing the median temperature, and the error bars representing the
min or max (IQR +/- IQR*1.5). Outliers are represented by points (e.g.,
MHW years if above the boxplot). For more information on interpretation
of boxplots see \url{https://r-graph-gallery.com/boxplot.html}}
\end{figure}

\begin{figure}
\centering
\includegraphics[width=1\textwidth,height=\textheight]{ESR_EBS/Figs/ACLIM_Holsman_Fig4.png}
\caption{Figure 4. Southern Bering Sea (SEBS) bottom water temperature
(oC) projected under two climate scenarios; high carbon mitigation via
Shared Socioeconomic Pathways (ssp126, left column); low carbon
mitigation, (ssp585, right column). Rows reflect the parent global model
simulation (miroc = MIROC ES2L, cesm = CESM2, gfdl = GFDL ESM4)
dynamically downscaled to a high resolution regional model (Bering10K
K20P19 30 layer ROMSNPZ model). Average modeled climatology from the
reference period (1980 -2013) of the historical run for each ESM is show
as the solid blue line; dashed lines represent +/- 1 standard deviation
of the mean.}
\end{figure}

\begin{figure}
\centering
\includegraphics[width=1\textwidth,height=\textheight]{ESR_EBS/Figs/ACLIM_Holsman_Fig5.png}
\caption{Figure 5. Northern Bering Sea (NEBS) bottom water temperature
(oC) projected under two climate scenarios; high carbon mitigation via
Shared Socioeconomic Pathways (ssp126, left column); low carbon
mitigation, (ssp585, right column). Rows reflect the parent global model
simulation (miroc = MIROC ES2L, cesm = CESM2, gfdl = GFDL ESM4)
dynamically downscaled to a high resolution regional model (Bering10K
K20P19 30 layer ROMSNPZ model). Average modeled climatology from the
reference period (1980 -2013) of the historical run for each ESM is show
as the solid blue line; dashed lines represent +/- 1 standard deviation
of the mean.}
\end{figure}

\end{document}
